%%%%%%%%%%%%%%%%%%%%%%%%%%%%%%%%%%%%%%%%%
% Short Sectioned Assignment
% LaTeX Template
% Version 1.0 (5/5/12)
%
% This template has been downloaded from:
% http://www.LaTeXTemplates.com
%
% Original author:
% Frits Wenneker (http://www.howtotex.com)
%
% License:
% CC BY-NC-SA 3.0 (http://creativecommons.org/licenses/by-nc-sa/3.0/)
%
%%%%%%%%%%%%%%%%%%%%%%%%%%%%%%%%%%%%%%%%%

%----------------------------------------------------------------------------------------
%	PACKAGES AND OTHER DOCUMENT CONFIGURATIONS
%----------------------------------------------------------------------------------------

\documentclass[paper=a4, fontsize=11pt]{scrartcl} % A4 paper and 11pt font size

\usepackage[T1]{fontenc} % Use 8-bit encoding that has 256 glyphs
\usepackage[adobe-utopia]{mathdesign}
%\usepackage{fourier} % Use the Adobe Utopia font for the document - comment this line to return to the LaTeX default
\usepackage[english]{babel} % English language/hyphenation
\usepackage{amsmath,amsfonts,amsthm} % Math packages

\usepackage{lipsum} % Used for inserting dummy 'Lorem ipsum' text into the template

\usepackage{sectsty} % Allows customizing section commands

\usepackage{braket}

\usepackage{graphicx}

\allsectionsfont{\centering \normalfont\scshape} % Make all sections centered, the default font and small caps


\usepackage{fancyhdr} % Custom headers and footers
\pagestyle{fancyplain} % Makes all pages in the document conform to the custom headers and footers
\fancyhead{} % No page header - if you want one, create it in the same way as the footers below
\fancyfoot[L]{} % Empty left footer
\fancyfoot[C]{} % Empty center footer
\fancyfoot[R]{\thepage} % Page numbering for right footer
\renewcommand{\headrulewidth}{0pt} % Remove header underlines
\renewcommand{\footrulewidth}{0pt} % Remove footer underlines
\setlength{\headheight}{13.6pt} % Customize the height of the header

\numberwithin{equation}{section} % Number equations within sections (i.e. 1.1, 1.2, 2.1, 2.2 instead of 1, 2, 3, 4)
\numberwithin{figure}{section} % Number figures within sections (i.e. 1.1, 1.2, 2.1, 2.2 instead of 1, 2, 3, 4)
\numberwithin{table}{section} % Number tables within sections (i.e. 1.1, 1.2, 2.1, 2.2 instead of 1, 2, 3, 4)

\setlength\parindent{0pt} % Removes all indentation from paragraphs - comment this line for an assignment with lots of text

%----------------------------------------------------------------------------------------
%	TITLE SECTION
%----------------------------------------------------------------------------------------

\newcommand{\horrule}[1]{\rule{\linewidth}{#1}} % Create horizontal rule command with 1 argument of height

\title{	
\normalfont \normalsize 
\textsc{Retention and Customer Experience} \\ [25pt] % Your us get downniversity, school and/or department name(s)
\horrule{0.5pt} \\[0.4cm] % Thin top horizontal rule
\huge Know Your Customers \\ % The assignment title
\horrule{2pt} \\[0.5cm] % Thick bottom horizontal rule
}

%\author{Brandon B. Miller} % Your name

\date{\normalsize\today} % Today's date or a custom date

\begin{document}

\maketitle % Print the title

\newpage

\section{Geographical Location}
The following is a map of the geographical distribution of some random variable over the contiguous United States. Zip codes are not defined as restricted to a specific spatial boundary, so data are binned by zipcode tabulation area (ZCTA) - shapes developed by the US Census Bureau to represent zip codes as closely as possible. The mapping used to defined the areas in this figure is sourced from the USCB webpage.  

\begin{figure}[h!]
\centering
\includegraphics[trim={3.75cm, 1cm, 0, 0.60cm}, scale=0.7]{sixteennine.eps}
\end{figure}

\begin{figure}[ht!]
\centering
\centering
\includegraphics[trim={0, 1cm, 0, 1cm}, width=.3\textwidth]{square.eps}\hfill
\includegraphics[trim={0, 1cm, 0, 1cm}, width=.3\textwidth]{square.eps}\hfill
\includegraphics[trim={0, 1cm, 0, 1cm}, width=.3\textwidth]{square.eps}
\caption{Of particular interest is the \textit{donut effect} in some of Allstate's most popular cities, possibly reflecting thinner car ownership in strictly urban areas.}
\end{figure}

\section{Customer Age}
The distribution of some andom variable years tied to some random company. The modes are roughly at the modes. The lower \textbf{FILL THIS IN} of the data is not shown. Note that these 

\begin{figure}[h!]
\centering
\includegraphics[trim={2cm, 1cm, 0, 0.60cm}, scale=0.6]{sixteennine.eps}
\includegraphics[trim={2cm, 1cm, 0, 0.60cm}, scale=0.6]{sixteennine.eps}
\end{figure}

\section{Customer Tenure}

  


%----------------------------------------------------------------------------------------
%	PROBLEM 1
%----------------------------------------------------------------------------------------

\end{document}
