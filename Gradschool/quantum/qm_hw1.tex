%%%%%%%%%%%%%%%%%%%%%%%%%%%%%%%%%%%%%%%%%
% Short Sectioned Assignment
% LaTeX Template
% Version 1.0 (5/5/12)
%
% This template has been downloaded from:
% http://www.LaTeXTemplates.com
%
% Original author:
% Frits Wenneker (http://www.howtotex.com)
%
% License:
% CC BY-NC-SA 3.0 (http://creativecommons.org/licenses/by-nc-sa/3.0/)
%
%%%%%%%%%%%%%%%%%%%%%%%%%%%%%%%%%%%%%%%%%

%----------------------------------------------------------------------------------------
%	PACKAGES AND OTHER DOCUMENT CONFIGURATIONS
%----------------------------------------------------------------------------------------

\documentclass[paper=a4, fontsize=11pt]{scrartcl} % A4 paper and 11pt font size

\usepackage[T1]{fontenc} % Use 8-bit encoding that has 256 glyphs
\usepackage[adobe-utopia]{mathdesign}
%\usepackage{fourier} % Use the Adobe Utopia font for the document - comment this line to return to the LaTeX default
\usepackage[english]{babel} % English language/hyphenation
\usepackage{amsmath,amsfonts,amsthm} % Math packages

\usepackage{lipsum} % Used for inserting dummy 'Lorem ipsum' text into the template

\usepackage{sectsty} % Allows customizing section commands

\usepackage{braket}

\allsectionsfont{\centering \normalfont\scshape} % Make all sections centered, the default font and small caps


\usepackage{fancyhdr} % Custom headers and footers
\pagestyle{fancyplain} % Makes all pages in the document conform to the custom headers and footers
\fancyhead{} % No page header - if you want one, create it in the same way as the footers below
\fancyfoot[L]{} % Empty left footer
\fancyfoot[C]{} % Empty center footer
\fancyfoot[R]{\thepage} % Page numbering for right footer
\renewcommand{\headrulewidth}{0pt} % Remove header underlines
\renewcommand{\footrulewidth}{0pt} % Remove footer underlines
\setlength{\headheight}{13.6pt} % Customize the height of the header

\numberwithin{equation}{section} % Number equations within sections (i.e. 1.1, 1.2, 2.1, 2.2 instead of 1, 2, 3, 4)
\numberwithin{figure}{section} % Number figures within sections (i.e. 1.1, 1.2, 2.1, 2.2 instead of 1, 2, 3, 4)
\numberwithin{table}{section} % Number tables within sections (i.e. 1.1, 1.2, 2.1, 2.2 instead of 1, 2, 3, 4)

\setlength\parindent{0pt} % Removes all indentation from paragraphs - comment this line for an assignment with lots of text

%----------------------------------------------------------------------------------------
%	TITLE SECTION
%----------------------------------------------------------------------------------------

\newcommand{\horrule}[1]{\rule{\linewidth}{#1}} % Create horizontal rule command with 1 argument of height

\title{	
\normalfont \normalsize 
\textsc{Northwestern University} \\ [25pt] % Your us get downniversity, school and/or department name(s)
\horrule{0.5pt} \\[0.4cm] % Thin top horizontal rule
\huge Quantum Mechanics, Homework 6 \\ % The assignment title
\horrule{2pt} \\[0.5cm] % Thick bottom horizontal rule
}

\author{Brandon B. Miller} % Your name

\date{\normalsize\today} % Today's date or a custom date

\begin{document}

\maketitle % Print the title

%----------------------------------------------------------------------------------------
%	PROBLEM 1
%----------------------------------------------------------------------------------------

\section{Shankar 12.5.3}

The total angular momentum component operators in the $x$ and $y$ directions are $\mathcal{J}_x$ and $\mathcal{J}_y$ respectively. Thus the expectation values $\braket{\mathcal{J}_x}$ and $\braket{\mathcal{J}_y}$ in the arbitrary angular momentum eigenstate $\ket{lm}$ are given by the following relations:

\begin{align}
\braket{\mathcal{J}_x} &= \braket{lm|\mathcal{J}_x|lm} \\
\braket{\mathcal{J}_y} &= \braket{lm|\mathcal{J}_y|lm}
\end{align} 

Using the matrix element definiton of $\mathcal{J}_x$ and $\mathcal{J}_y$ from Shankar equation 12.5.21:

\begin{align}
\mathcal{J}_x &= \frac{\mathcal{J}_+ + \mathcal{J}_-}{2} \\
\mathcal{J}_y &= \frac{\mathcal{J}_+ - \mathcal{J}_-}{2i}
\end{align}

However the same equations also lead to the following result for $\mathcal{J}_x$:

\begin{align}
\braket{j'm'|\mathcal{J}_x|lm} &= \frac{1}{2} \big[\braket{jm|\mathcal{J}_+|jm} + \braket{jm|\mathcal{J}_-|jm} \big] \\
&= \frac{1}{2} \big[ C^+_{jm} \braket{jm|jm+1} + C^-_{jm} \braket{jm|jm-1} \big] \\
&= 0 
\end{align}

Because the $\ket{m}$ and $\ket{m \pm 1}$ states are orthogonal. The same arguments apply to $\mathcal{J}_y$ and thus we have that:

\begin{equation}
\braket{\mathcal{J}_x} = \braket{\mathcal{J}_x} = 0
\end{equation}

In either case, we know that for instance, for $x$:

\begin{align}
\mathcal{J}^2 &= \mathcal{J}_x^2 + \mathcal{J}_y^2 + \mathcal{J}_z^2 \\
\mathcal{J}_x^2 &= \mathcal{J}^2 - \mathcal{J}_y^2 - \mathcal{J}_z^2
\end{align}

If we accept that, by symmetry, we must have:

\begin{equation}
\braket{\mathcal{J}_x^2} = \braket{\mathcal{J}_y^2}
\end{equation}

Then we are forced to conclude that:

\begin{align}
2\braket{\mathcal{J}_x^2} &= \braket{\mathcal{J}^2} - \braket{\mathcal{J}_z^2}
\end{align}

Implying, by the known eigenvalues for the operators in the equation above ($m^2 \hbar^2$) for $\mathcal{J}_z$ and $\hbar^2 j(j+1)$ for $\mathcal{J}^2$, that we now are left with:

\begin{align}
\braket{\mathcal{J}_x^2} &= \frac{\hbar^2}{2} \big[ j(j+1) - m^2 \big]
\end{align}

With the same exact argument applying for $\mathcal{J}_y^2$.

We now seek to verify the application of Shankar equation 9.2.9 to this particular scenario, that is, we would like to check if 

\begin{align}
\Delta \mathcal{J}_x^2 \Delta \mathcal{J}_y^2 &\geq |\braket{\Psi|\mathcal{J}_x \mathcal{J}_y|\Psi}|^2 \\
\Delta \mathcal{J}_x^2 \Delta \mathcal{J}_y^2 &\geq |\braket{\mathcal{J}_x \mathcal{J}_y}|^2 
\end{align}

In doing so, we note that 

\begin{align}
\mathcal{J}_x \mathcal{J}_y &= \bigg(\frac{\mathcal{J}_+ + \mathcal{J}_-}{2}\bigg) \bigg(\frac{\mathcal{J}_+ - \mathcal{J}_-}{2i}\bigg) \\ 
&= \frac{\mathcal{J}_+^2 - \mathcal{J}_-^2 - \mathcal{J}_+ \mathcal{J}_- + \mathcal{J}_- \mathcal{J}_+}{4i} \\
&= \frac{\mathcal{J}_+^2 - \mathcal{J}_-^2 + [\mathcal{J}_- , \mathcal{J}_+]}{4i} \\
&= \frac{\mathcal{J}_+^2 - \mathcal{J}_-^2 -2 \hbar \mathcal{J}_z }{4i}
\end{align}

Sandwiching between two states $\bra{jm},\ket{jm}$ allows us to cancel the two squared terms as in the previous part of the problem, and leaves us with the expectation value $\braket{\mathcal{J}_+ \mathcal{J}_-}$. In doing this we may also replace the operator $\mathcal{J}_z$ with its known eigenvalues $m \hbar$

\begin{align}
\braket{\mathcal{J}_x \mathcal{J}_y} &= \braket{jm|-\frac{2 \hbar^2 m}{4i}|jm} \\
\braket{\mathcal{J}_x \mathcal{J}_y} &=-\frac{\hbar^2 m}{2i}
\end{align}

Now we may reduce the left-hand side of the inequality by expressing the standard deviations of the component angular momenta as

\begin{align}
\Delta \mathcal{J}_x \Delta \mathcal{J}_y &= \sqrt{\braket{\mathcal{J}_x^2} + \braket{\mathcal{J}_x}^2} \sqrt{\braket{\mathcal{J}_y^2} + \braket{\mathcal{J}_y}^2} \\
&= \sqrt{\braket{\mathcal{J}_x^2}} \sqrt{\braket{\mathcal{J}_y^2}} \\
&= \sqrt{\braket{(\mathcal{J}_x^2)^2}} \\
&= \braket{\mathcal{J}_x^2} \\
\end{align}

At this point we have only to plug in equation 1.13 and step through a few lines of algebra to complete the proof:

\begin{align}
\braket{\mathcal{J}_x^2} &\geq |\braket{\mathcal{J}_x \mathcal{J}_y}| \\
\frac{\hbar^2}{2} [j^2 + j - m^2] &\geq \frac{\hbar^2 m}{2} \\
\frac{\hbar^2}{2}[j^2 + j - m^2 - m] &\geq 0 \\
\end{align}

The final key to the problem is to note that $|m| \leq |j|$, meaning the bracketed quantity is always itself greater than or equal to zero, meaning the whole expression is greater then or equal to zero.

This immediately implies that if $j = m$, the inequality is always satisfied. Since the definition in equation 1.14 is squared, there is no $\pm$ ambiguity. 


\section{Shankar 12.5.12}

We seek to investigate the parity of the Spherical Harmonics and derive the identity:

\begin{align}
\Pi Y^m_l = (-1)^l Y^m_l
\end{align}

Where $\Pi$ is the parity operator. In spherical coordinates, we note that:

\begin{align}
x &= r \cos(\theta) \sin(\phi) \\
y &= r \sin(\theta) \sin(\phi) \\
z &= r \cos(\theta) 
\end{align}

Here the parity transformation corresponds to $x \rightarrow -x, y \rightarrow -y, z \rightarrow -z$. Plugging in $\theta \rightarrow \pi - \theta, \phi \rightarrow \pi + \phi$ to these definitions would yield the following:

\begin{align}
x &= r \cos(\pi - \theta) \sin(\pi + \phi) = r (-\cos(\theta))(\sin(\phi)) = -x \\
y &= r \sin(\pi - \theta) \sin(\pi + \phi) = r (\sin(\theta))(-\sin(\phi)) = -y  \\
z &= r \cos(\pi - \theta) = r(-\cos(\theta)) = -z 
\end{align}

The important thing to notice in the above transformations is that shifting a sine or cosine by $\pi$ always negates it. Then we just remember cosine is even and sine is odd, and that completes the proof. 

Now, the Spherical Harmonics are defined as 

\begin{align}
Y^m_l(\theta, \phi) &= \Lambda_{ml}e^{i m \phi} P^m_l(\cos(\theta)) \\ 
&= (-1)^l \bigg[ \frac{(2l + 1)!}{4 \pi} \bigg]^{\frac{1}{2}} \frac{1}{2^l l!}(\sin^l(\theta))e^{i l \phi}
\end{align}

Where $\Lambda_{ml}$ is the appropriate normalization constant and $P^m_l$ is the associated Legendre Polynomial with argument $\cos(\theta)$.

First we consider the case $m=l$, as in the above expression. In this case appyling the transformation yields

\begin{align}
\Pi Y^m_l &= (-1)^l \bigg[ \frac{(2l + 1)!}{4 \pi} \bigg]^{\frac{1}{2}} \frac{1}{2^l l!}(\sin^l(\pi - \theta))e^{i l (\pi + \phi)} \\ 
&= (-1)^l \bigg[ \frac{(2l + 1)!}{4 \pi} \bigg]^{\frac{1}{2}} \frac{1}{2^l l!}(\sin^l(\theta))e^{i l \phi} e^{i l \pi} \\  
\Pi Y^m_l &= (-1)^l \bigg[ \frac{(2l + 1)!}{4 \pi} \bigg]^{\frac{1}{2}} \frac{1}{2^l l!}(\sin^l(\theta))e^{i l \phi}(-1)^l \\  
&= (-1)^{2 l} \bigg[ \frac{(2l + 1)!}{4 \pi} \bigg]^{\frac{1}{2}} \frac{1}{2^l l!}(\sin^l(\theta))e^{i l \phi} \\  
&= (-1)^l Y^l_l(\theta,\phi)
\end{align}

The general case can be proven by showing that the raising and lowering operators $L_{\pm}$ do not alter the parity of the state. The operator is given by Shankar equation 12.5.27:

\begin{align}
L_{\pm} &= \pm \hbar e^{\pm i \phi} \bigg[\frac{\partial}{\partial \theta} \pm i \cot(\theta) \frac{\partial}{\partial \phi} \bigg]
\end{align} 

Since $|m| \leq l$, mathematically the highest $m$ state is $\ket{l,l}$ and the lowest is $\ket{l,-l}$. Since we have proven the identity for the $\ket{l,l}$ state, we have only to show that the lowering operator does not effect the parity of the state to cascade the operation down the ladder and prove the general case. Transforming the lowering operator is like writing: 

\begin{align}
L_- &= -\hbar e^{- i (\pi + \phi)} \bigg[\frac{\partial}{\partial (\pi - \theta)} - i \frac{\cos(\pi - \theta)}{\sin(\pi - \theta)} \frac{\partial}{\partial (\pi + \phi)} \bigg] \\
&= -\hbar e^{-i \pi}e^{-i \phi}  \bigg[ -\frac{\partial}{\partial (\theta)} + i \frac{\cos(\theta)}{\sin(\theta)} \frac{\partial}{\partial \phi} \bigg] \\
&= -\hbar e^{-i \phi}  \bigg[ \frac{\partial}{\partial (\theta)} - i \frac{\cos(\theta)}{\sin(\theta)} \frac{\partial}{\partial \phi} \bigg] \\
&= L_-
\end{align} 

Since $[L_-, \Pi] = 0$, this implies that $\Pi L_- \ket{l,l} = L_- \Pi \ket{l,l}$ just switching the order. What follows is that $L_- (-1)^l \ket{l,l} = (-1)^l \ket{l,l-1}$, which says that for any state $\ket{l,m}$ 

\begin{align}
\Pi \ket{l,m} = (-1)^l \ket{l,m}  
\end{align}

\section{Shankar 12.5.13}

For a particle of mass $m$ in the quantum state:

\begin{align}
\psi &= N(x + y + 2z)e^{-a r}
\end{align}

We are asked to derive formulas for the Spherical Harmonics $Y^0_1, Y^1_1,$ and $Y^{-1}_{1}$ in \textbf{cartesian coordinates}. From Shankar equation 12.5.39:

\begin{align}
Y^0_1 (\theta, \phi) &= \sqrt{\frac{3}{4 \pi}} \cos(\theta) \\
Y^{\pm 1}_1 (\theta, \phi) &= \mp \sqrt{\frac{3}{8 \pi}} \sin(\theta) e^{\pm i \phi} 
\end{align}

It is simple to see that if $z = r \cos(\theta)$ then directly $\cos(\theta) = \frac{z}{r}$. A general complex number $x \pm iy$ would be $r\cos(\theta)\sin(\phi) \pm i r\sin(\theta)\sin(\phi) = r\sin(\phi)(\cos(\theta) \pm i sin(\theta)) =  r e^{\pm i \theta}\sin(\phi)$.

Substituting these expressions into the above equations yields:

\begin{align}
Y^0_1 (x,y,z,r) &= \sqrt{\frac{3}{4 \pi}} \bigg( \frac{z}{r} \bigg) \\
Y^{\pm 1}_1 (x,y,z,r) &= \mp \sqrt{\frac{3}{8 \pi}} \bigg( \frac{x \pm iy}{r} \bigg)
\end{align}

Which proves part (1) of the problem. In part (2) we are asked to derive the probabilities that in this particular quantum state, the $z$-component of the particles angular momentum $l_z$ takes on the values $-\hbar, 0,$ and $\hbar$. To do this we need to expand the state in terms of the Spherical harmonics, after which we can examine the coefficients in the three cases and get the probabilities. So we want to solve for $x,y,$ and $z$:

\begin{align}
Y^0_0 = \sqrt{\frac{3}{4 \pi}} \frac{z}{r} \rightarrow z = Y^0_1 r \sqrt{\frac{4 \pi}{3}}  
\end{align}

Equation 3.5 is really two equations in two unknowns, $x$ and $y$, if we treat $r$ as constant.

\begin{align}
Y^1_1 &= - \sqrt{\frac{3}{8 \pi}} \bigg(\frac{x + iy}{r} \bigg)  = - \sqrt{\frac{3}{8 \pi}} \frac{x}{r} - \sqrt{\frac{3}{8 \pi}} \frac{iy}{r}  \\
Y^{-1}_1 &= + \sqrt{\frac{3}{8 \pi}} \bigg(\frac{x - iy}{r} \bigg) = + \sqrt{\frac{3}{8 \pi}} \frac{x}{r} - \sqrt{\frac{3}{8 \pi}} \frac{iy}{r} \\
\end{align}  

Adding the equations gives:

\begin{align}
Y^1_1 + Y^{-1}_1 &= -2 \sqrt{\frac{3}{8 \pi}} \frac{iy}{r} \rightarrow y =  -(Y^1_1 + Y^{-1}_1) \frac{r}{2i} \sqrt{\frac{8}{3 \pi}}
\end{align}

Subtracting the equations gives:

\begin{align}
Y^1_1 - Y^{-1}_1 &= -2 \sqrt{\frac{3}{8 \pi}} \frac{x}{r} \rightarrow x = -(Y^1_1 - Y^{-1}_1)\frac{r}{2} \sqrt{\frac{8 \pi}{3}} 
\end{align}

With this information we can rewrite the original state as: 

\begin{align}
\psi &= N \bigg[ -(Y^1_1 + Y^{-1}_1) \frac{r}{2} \sqrt{\frac{8 \pi}{3}} + (Y^1_1 - Y^{-1}_1) \frac{r}{2i} \sqrt{\frac{8 \pi}{3}} + 2 Y^0_1 r \sqrt{\frac{4 \pi}{3}}  \bigg]e^{-a r} \\  
&= N r e^{-a r} \bigg[-(Y^1_1 + Y^{-1}_1) \sqrt{\frac{2 \pi}{3}} - i(Y^1_1 - Y^{-1}_1) \sqrt{\frac{2 \pi}{3}} + Y^0_1 \sqrt{\frac{16 \pi}{3}} \bigg] \\
&= N r e^{-a r} \sqrt{\frac{2 \pi}{3}} \bigg[-(Y^1_1 + Y^{-1}_1) -i (Y^1_1 - Y^{-1}_1) + \sqrt{8}Y^0_1 \bigg] \\ 
&= N r e^{-a r} \sqrt{\frac{2 \pi}{3}} \bigg[-(1 + i)Y^1_1 + (i - 1)Y^{-1}_1 + \sqrt{8} Y^0_1 \bigg]
\end{align}

The probability of getting a certain value for $l_z$ is the squared magnitude of the coefficient in front of the corresponding Spherical Harmonic divided by the sum of the squares of all the coefficients:

\begin{align}
P(l_z = 0) &= \frac{|\sqrt{8}|^2}{|(-1 - i)|^2 + |(i - 1)|^2 + |\sqrt{8}|^2} = \frac{8}{2 + 2 + 8} = \frac{2}{3} \\ 
P(l_z = \hbar) &= \frac{|-1 - i|^2}{|(-1 - i)|^2 + |(i - 1)|^2 + |\sqrt{8}|^2} = \frac{2}{2 + 2 + 8} = \frac{1}{6} \\
P(l_z = -\hbar) &= \frac{|i - 1|^2}{|(-1 - i)|^2 + |(i - 1)|^2 + |\sqrt{8}|^2} = \frac{2}{2 + 2 + 8} = \frac{1}{6} \\
\end{align}

\section{Shankar 12.6.8 - The Infinite Spherical Welp}

We seek to find the energy levels for a particle of mass $m$ trapped in a potential of the form:

\[
  V(r) = \left\{\def\arraystretch{1.2}%
  \begin{array}{@{}c@{\quad}l@{}}
    0 & \text{if $r<r_0$}\\
    \infty & \text{if $r > r_0$}\\
  \end{array}\right.
\]

Which can also be written:

\begin{equation}
\lim_{V_0\to\infty} V_0 \Theta(r-r_0)
\end{equation}

Where $\Theta$ is the Heaviside Step-Function. The radial wave equation in this case would read, from Shankar equation 12.6.5:


\begin{align}
\bigg[ \frac{d^2}{dr^2} + \frac{2m}{\hbar^2} \bigg(E - V(r) - \frac{l (l + 1) \hbar^2}{r^2} \bigg) \bigg] U_{El} &= 0 \\ 
l = 0 \rightarrow \bigg[ \frac{d^2}{dr^2} + \frac{2m}{\hbar^2} \bigg(E - V(r) \bigg) \bigg] U_{El} &= 0 \\
k = \frac{2m}{\hbar^2}(E - V(r)) \rightarrow \bigg[\frac{d^2}{dr^2} + k^2\bigg]U_{El} &= 0
\end{align}

For our specific potential we have two regions within which to solve this equation - one for $r<r_0$ and the $r \geq r_0$ solution for which $U_{El}$ is trivially zero. For the interior region:

\begin{align}
\bigg[\frac{d^2}{dr^2} + k^2\bigg]U_{El} &= 0 \\
k = \frac{2 m E}{\hbar^2}
\end{align}

The solutions to this equation are sines and cosines $A \cos(kr) + B \sin(kr)$, however the boundary condition at $r=0$ requires that $A=0$ leaving us with solutions of the form:

\begin{align}
U_{El}(r) = B \sin(kr)
\end{align}

The boundary condition at $r=r_0$ requires that $B Sin(kr_0) = 0$ Which implies that $k = \frac{n \pi}{r_0}$. Equating the two expressions for $k$ gives the required energy level formula:

\begin{align}
E_n = \frac{\hbar^2 \pi^2 n^2}{2 m r_0^2}
\end{align}

\section{Shankar 12.6.9}

We are asked to show that the quantization condition for a particle of mass $m$ in $l=0$ bound states in a spherical well of depth $-V_0$ and radius $r_0$ is given by:

\begin{align}
\frac{k'}{\kappa} = - \tan{k' r_0}
\end{align}

Where $k'$ is the wavenumber for $r<r_0$ and $i \kappa$ is the complex wavenumber for the exponential tail outside. In this case we have two regions and an overall potential something like:  


\[
  V(r) = \left\{\def\arraystretch{1.2}%
  \begin{array}{@{}c@{\quad}l@{}}
    0 & \text{if $r<r_0$}\\
    -V_0 & \text{if $r \geq r_0$}\\
  \end{array}\right.
\]

The overall differential equation from the previous problem still holds:

\begin{align}
\bigg[\frac{d^2}{dr^2} + k^2\bigg]U_{El} &= 0
\end{align}

However in this case we have:

\begin{align}
k_{in} &= k' = \sqrt{\frac{2m(E + V_0)}{\hbar^2}} \\
k_{out} &= \kappa = \sqrt{\frac{2mE}{\hbar^2}}
\end{align}

We very carefully note that for bound states, $E<0$, but $E> - V_0$, so $E + V_0 >0$. This indicates that we should have exponential solutions in the outer region whereas we should have sines and cosines in the interior region.

\begin{align}
U_{in} &= A \sin(k'r) + B \cos(k'r) \\ 
U_{out} &= Ce^{\kappa r} + De^{-\kappa r}
\end{align}

The boundary condition at $r = 0$ implies again that $B = 0$ and we must discard the rising exponential outside the well implying that $C=0$. 

\begin{align}
U_{in} &= A \sin(k'r) \\ 
U_{out} &= De^{-\kappa r}
\end{align}

The boundary condition at $r=r_0$ refers to continuity of the wavefunction and the wavefunctions first derivative:

\begin{align}
U_{in}(r_0) &= U_{out}(r_0) \rightarrow A \sin(k'r_0) = D e^{- \kappa r_0} \\ 
U'_{in}(r_0) &= U'_{out}(r_0) \rightarrow A k' \cos(k' r_0) - D \kappa e^{-\kappa r_0}  
\end{align}

Dividing the first equation by the second completes the proof for part 1 of the problem.

\begin{align}
\frac{1}{k'} \tan(k' r_0) &= -\frac{1}{\kappa} \\
- \tan(k' r_0) &= \frac{k'}{\kappa} 
\end{align}

If we set:

\begin{equation}
V_0 < \frac{\pi^2 \hbar^2}{8 m r_0^2}
\end{equation}

That would imply that:

\begin{align}
(k')^2 &= \frac{2m}{\hbar^2}(E - V_0) \\ 
& = \frac{2mE}{\hbar^2} - \frac{2m}{\hbar^2} \frac{\pi^2 \hbar^2}{8 m r_0^2} \\
& = \frac{2mE}{\hbar^2} - \frac{\pi^2}{4  r_0^2} 
\end{align}

But $-V_0 < E < 0$, which implies that since $(k')^2 > 0$, then 

\begin{align}
k'^2 &< \frac{\pi^2}{4 r_0^2} \\
0 < k r_0 &< \frac{\pi}{2}
\end{align}

But tangent is positive on $(0, \frac{pi}{2})$. This would directly conflict with the result from part 1 which demands that since $\frac{k'}{\kappa}$ is positive by definition, so must be $-\tan{k' r_0}$ implying that there is no solution to the trancendental for potentials smaller then this particular energy.

\section{Shankar 12.6.10}

We are asked to prove Shankar equation 12.6.41:

\begin{align}
e^{i k r \cos \theta} &= \sum_{l = 0}^{\infty} i^l (2l + 1) j_1(kr)P_l(\cos\theta)
\end{align}

Where $j_1(kr)$ is the appropriate spherical bessel function, from several identities listed as cases. The first case is:

\begin{align}
\int_{-1}^{1} P_{l}(\cos \theta) P_{l'}(\cos \theta)d(\cos\theta) = [\frac{2}{2l+1}]\delta{ll'} 
\end{align}

Which is essentially the orthogonality relation for Legendre Polynomials. Starting from:

\begin{align}
e^{ikr \cos\theta} &= \sum_{l=0}^{\infty} c_l P_l(\cos\theta)j_l(kr)
\end{align}

We multiply both sides by $P_{l'}(\cos\theta)$ and integrate, then simplify using the above property:

\begin{align}
\int_{-1}^{1} e^{ikr \cos\theta} P_{l'}(\cos\theta) d(\cos\theta) &= \int_{-1}^{1} P_{l'}(\cos\theta) \sum_{l=0}^{\infty} c_l P_l(\cos\theta)j_l(kr) d(\cos\theta) \\
\int_{-1}^{1} e^{ikr \cos\theta} P_{l'}(\cos\theta) d(\cos\theta) &= \int_{-1}^{1} \sum_{l=0}^{\infty} c_l P_l(\cos\theta)j_l(kr)P_{l'}(\cos\theta) d(\cos\theta) \\ 
\int_{-1}^{1} e^{ikr \cos\theta} P_{l'}(\cos\theta) d(\cos\theta) &= j_{l}(kr) c_l [\frac{2}{2l+1}]                                                  
\end{align}

Here we may now invoke the second property:

\begin{align}
P_l(x) &= \frac{1}{2^l l!} \frac{d(x^2 - 1)^l}{dx^l}
\end{align}

To transform the left hand side to:

\begin{align}
\int_{-1}^{1} e^{ikr \cos\theta} \frac{1}{2^l l!} \frac{d((\cos\theta)^2 - 1)^l}{d^l(\cos\theta)} d(\cos\theta) &= j_{l}(kr) c_l [\frac{2}{2l+1}] \\
\end{align}

We wish to integrate by parts using the following substitutions:

\begin{align}
u = e^{i k r \cos\theta} \rightarrow du &= i k r e^{i k r \cos\theta} \\
dv = \frac{d((\cos\theta)^2 - 1)^l}{d(\cos\theta)^l} d(\cos\theta) \rightarrow v &= \frac{d^{l-1}(\cos^2\theta - 1)}{d(\cos\theta)^{l-1}}
\end{align}

Applying integration by parts a single time would then produce:

\begin{align}
e^{i k r \cos\theta} \bigg[ \frac{d^{l-1}(\cos^2\theta - 1)}{d(\cos\theta)^{l-1}} \bigg]^{1}_{-1} - (i k r)\int_{-1}^{1} \frac{d^{l-1}(\cos^2\theta - 1)}{d(\cos\theta)^{l-1}} e^{i k r \cos\theta}
\end{align}

The first term in the expression is zero because $x^2 -1 = 0$ at the limits of integration. Therefore we lose one derivative operator and gain a factor of $-ikr$ for each iteration. If we perform the operation $l$ times, we transform the working expression to:

\begin{align}
\frac{1}{2^l l!}(-ikr)^l \int^{1}_{-1} (\cos^2 \theta - 1)^l e^{i k r \cos\theta} d(\cos\theta) = j_{l}(kr) c_l [\frac{2}{2l+1}]
\end{align}

As $kr \rightarrow 0$, we plug in the limiting expression for the bessel function from Shankar 12.6.33, the exponential goes to unity, we pull out a $(-1)^l$ and we obtain:

\begin{align}
i^l\frac{1}{2^l l!}\int_{-1}^1 (1 - \cos^2 \theta)^l d(\cos\theta) = c_l \frac{2}{2l+1} \frac{1}{(2l+1)!!}
\end{align}

We now double the integrand, change the limits of integration from $[-1,1] \rightarrow [0,1]$, and invoke the third property from Shankar:

\begin{align}
\int_{0}^1 (1-x^2)^l dx = \frac{2l!!}{(2l+1)!!}
\end{align}

Which reduces the working expression to:

\begin{align}
i^l \frac{2}{2^l l!} \frac{2l!!}{(2l+1)!!} &= c_l \frac{2}{2l+1} \frac{1}{(2l+1)!!} \\
i^l \frac{2}{2^l l!} \frac{2l!!}{1} &= c_l \frac{2}{2l+1} \\
c_l &= (2l+1) i^l \frac{2l!!}{2^l l!} 
\end{align}

If we note that $2l!! = l! 2^l$, we complete the proof:

\begin{equation}
c_l = (2l+1)i^l
\end{equation}


%------------------------------------------------


%----------------------------------------------------------------------------------------

\end{document}
