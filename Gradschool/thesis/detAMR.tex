%%%%%%%%%%%%%%%%%%%%%%%%%%%%%%%%%%%%%%%%%
% Short Sectioned Assignment
% LaTeX Template
% Version 1.0 (5/5/12)
%
% This template has been downloaded from:
% http://www.LaTeXTemplates.com
%
% Original author:
% Frits Wenneker (http://www.howtotex.com)
%
% License:
% CC BY-NC-SA 3.0 (http://creativecommons.org/licenses/by-nc-sa/3.0/)
%
%%%%%%%%%%%%%%%%%%%%%%%%%%%%%%%%%%%%%%%%%

%----------------------------------------------------------------------------------------
%	PACKAGES AND OTHER DOCUMENT CONFIGURATIONS
%----------------------------------------------------------------------------------------

\documentclass[paper=a4, fontsize=11pt]{scrartcl} % A4 paper and 11pt font size

\usepackage[T1]{fontenc} % Use 8-bit encoding that has 256 glyphs
\usepackage[adobe-utopia]{mathdesign}
%\usepackage{fourier} % Use the Adobe Utopia font for the document - comment this line to return to the LaTeX default
\usepackage[english]{babel} % English language/hyphenation
\usepackage{amsmath,amsfonts,amsthm} % Math packages

\usepackage{lipsum} % Used for inserting dummy 'Lorem ipsum' text into the template

\usepackage{sectsty} % Allows customizing section commands

\usepackage{braket}

\allsectionsfont{\centering \normalfont\scshape} % Make all sections centered, the default font and small caps


\usepackage{fancyhdr} % Custom headers and footers
\pagestyle{fancyplain} % Makes all pages in the document conform to the custom headers and footers
\fancyhead{} % No page header - if you want one, create it in the same way as the footers below
\fancyfoot[L]{} % Empty left footer
\fancyfoot[C]{} % Empty center footer
\fancyfoot[R]{\thepage} % Page numbering for right footer
\renewcommand{\headrulewidth}{0pt} % Remove header underlines
\renewcommand{\footrulewidth}{0pt} % Remove footer underlines
\setlength{\headheight}{13.6pt} % Customize the height of the header

\numberwithin{equation}{section} % Number equations within sections (i.e. 1.1, 1.2, 2.1, 2.2 instead of 1, 2, 3, 4)
\numberwithin{figure}{section} % Number figures within sections (i.e. 1.1, 1.2, 2.1, 2.2 instead of 1, 2, 3, 4)
\numberwithin{table}{section} % Number tables within sections (i.e. 1.1, 1.2, 2.1, 2.2 instead of 1, 2, 3, 4)

\setlength\parindent{0pt} % Removes all indentation from paragraphs - comment this line for an assignment with lots of text

%----------------------------------------------------------------------------------------
%	TITLE SECTION
%----------------------------------------------------------------------------------------

\newcommand{\horrule}[1]{\rule{\linewidth}{#1}} % Create horizontal rule command with 1 argument of height

\title{	
\normalfont \normalsize 
\textsc{Northwestern University} \\ [25pt] % Your us get downniversity, school and/or department name(s)
\horrule{0.5pt} \\[0.4cm] % Thin top horizontal rule
\huge Detector AM Response Vectorization \\ % The assignment title
\horrule{2pt} \\[0.5cm] % Thick bottom horizontal rule
}

\author{Brandon B. Miller} % Your name

\date{\normalsize\today} % Today's date or a custom date

\begin{document}

\maketitle % Print the title

\section{ComputeDetAMResponse}
The output of this function are the components of what will become a complex number, $F_{+}$ and $F_{\times}$. These are long sums of different terms that, based on the current code, look like this:

\begin{align}
F_{+} &= [R_{0,0}X_0 + R_{0,1}X_1 + R_{0,2}X_2]X_0 - [R_{0,0}Y_0 + R_{0,1}Y_1 + R_{0,2}Y_2]Y_0 \\
&+ [R_{1,0}X_0 + R_{1,1}X_1 + R_{1,2}X_2]X_1 - [R_{1,0}Y_0 + R_{1,1}Y_1 + R_{1,2}Y_2]Y_1 \\
&+ [R_{2,0}X_0 + R_{2,1}X_1 + R_{2,2}X_2]X_2 - [R_{2,0}Y_0 + R_{2,1}Y_1 + R_{2,2}Y_2]Y_2
\end{align}

Collecting all the positive terms of this expression gives
\begin{align}
F_{+} &= R_{0,0}X_0X_0 + R_{0,1}X_0X_1 + R_{0,2}X_0X_2 \\
&+ R_{0,0}X_0X_0 + R_{0,1}X_0X_1 + R_{0,2}X_0X_2 \\ 
&+ R_{0,0}X_0X_0 + R_{0,1}X_0X_1 + R_{0,2}X_0X_2 
\end{align}
At which point we see that the same result is computable as an outer product of the vector $\vec{X}$ using

\begin{align}
F_{+} &= \vec{X}\mathbf{R}\vec{X} - \vec{Y}\mathbf{R}\vec{Y}
\end{align}

And the complex part of the gravitational wave is thus

\begin{align}
F_{\times} &= \vec{X}\mathbf{R}\vec{Y} + \vec{Y}\mathbf{R}\vec{X}
\end{align}

One a sample-to-sample basis, the numbers that vary are the components of the vectors $\vec{X}$ and $\vec{Y}$. Thus we need a function that takes as input a \textit{list} vectors (or \textit{vector}) of vectors and produces a vector with the right components as output. To that end we define the tensor $X^{i}_{j}$ where 

\begin{align}
X^i &= 
\begin{bmatrix}
X^{i}_0 \\
X^{i}_1 \\
X^{i}_2
\end{bmatrix}
\end{align}

As well as the tensor $R^{i}_{jk}$ where

\begin{align}
R^{i} &= 
\begin{bmatrix}
R^{i}_{00} & R^{i}_{01} & R^{i}_{02} \\ 
R^{i}_{10} & R^{i}_{11} & R^{i}_{12} \\ 
R^{i}_{20} & R^{i}_{21} & R^{i}_{22}  
\end{bmatrix}
\end{align} 

The tensor $X$ is like a stack of all the different possible $\vec{X}$ coming out of the page. The Tensor $R$ is like $n$ copies of the matrix $\mathbf{R}$ stacked on top of each other. In this way the desired vector is obtainable with the tensor contraction 

\begin{align}
F^{i}_{+} &= X^{lm}R^{i}_{lj}X^{j}_{m} - Y^{lm}R^{i}_{lj}Y^{j}_{m} \\ 
F^{i}_{\times} &= X^{lm}R^{i}_{lj}Y^{j}_{m} + Y^{lm}R^{i}_{lj}X^{j}_{m} \\ 
\end{align}

\end{document}
