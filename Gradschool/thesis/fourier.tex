%%%%%%%%%%%%%%%%%%%%%%%%%%%%%%%%%%%%%%%%%
% Short Sectioned Assignment
% LaTeX Template
% Version 1.0 (5/5/12)
%
% This template has been downloaded from:
% http://www.LaTeXTemplates.com
%
% Original author:
% Frits Wenneker (http://www.howtotex.com)
%
% License:
% CC BY-NC-SA 3.0 (http://creativecommons.org/licenses/by-nc-sa/3.0/)
%
%%%%%%%%%%%%%%%%%%%%%%%%%%%%%%%%%%%%%%%%%

%----------------------------------------------------------------------------------------
%	PACKAGES AND OTHER DOCUMENT CONFIGURATIONS
%----------------------------------------------------------------------------------------

\documentclass[paper=a4, fontsize=11pt]{scrartcl} % A4 paper and 11pt font size

\usepackage[T1]{fontenc} % Use 8-bit encoding that has 256 glyphs
\usepackage[adobe-utopia]{mathdesign}
%\usepackage{fourier} % Use the Adobe Utopia font for the document - comment this line to return to the LaTeX default
\usepackage[english]{babel} % English language/hyphenation
\usepackage{amsmath,amsfonts,amsthm} % Math packages

\usepackage{lipsum} % Used for inserting dummy 'Lorem ipsum' text into the template

\usepackage{sectsty} % Allows customizing section commands

\usepackage{braket}

\allsectionsfont{\centering \normalfont\scshape} % Make all sections centered, the default font and small caps


\usepackage{fancyhdr} % Custom headers and footers
\pagestyle{fancyplain} % Makes all pages in the document conform to the custom headers and footers
\fancyhead{} % No page header - if you want one, create it in the same way as the footers below
\fancyfoot[L]{} % Empty left footer
\fancyfoot[C]{} % Empty center footer
\fancyfoot[R]{\thepage} % Page numbering for right footer
\renewcommand{\headrulewidth}{0pt} % Remove header underlines
\renewcommand{\footrulewidth}{0pt} % Remove footer underlines
\setlength{\headheight}{13.6pt} % Customize the height of the header

\numberwithin{equation}{section} % Number equations within sections (i.e. 1.1, 1.2, 2.1, 2.2 instead of 1, 2, 3, 4)
\numberwithin{figure}{section} % Number figures within sections (i.e. 1.1, 1.2, 2.1, 2.2 instead of 1, 2, 3, 4)
\numberwithin{table}{section} % Number tables within sections (i.e. 1.1, 1.2, 2.1, 2.2 instead of 1, 2, 3, 4)

\setlength\parindent{0pt} % Removes all indentation from paragraphs - comment this line for an assignment with lots of text

%----------------------------------------------------------------------------------------
%	TITLE SECTION
%----------------------------------------------------------------------------------------

\newcommand{\horrule}[1]{\rule{\linewidth}{#1}} % Create horizontal rule command with 1 argument of height

\title{	
\normalfont \normalsize 
\textsc{Northwestern University} \\ [25pt] % Your us get downniversity, school and/or department name(s)
\horrule{0.5pt} \\[0.4cm] % Thin top horizontal rule
\huge Fourier Transforms, U and V Matrices \\ % The assignment title
\horrule{2pt} \\[0.5cm] % Thick bottom horizontal rule
}

\author{Brandon B. Miller} % Your name

\date{\normalsize\today} % Today's date or a custom date

\begin{document}

\maketitle % Print the title

The fourier transform $\hat{h}(\omega)$ of a real or complex function of time $h(t)$ is defined as

\begin{align}
\hat{h}(\omega) &\equiv \int_{-\infty}^{\infty} h(t)e^{-i \omega t} dt
\end{align}

For purely real $h(t)$ the following relation holds. Note that the complex conjugation and reversal operations commute:

\begin{align}
[\hat{h}(\omega)]^{*} &= \int_{-\infty}^{\infty}h(t)e^{i \omega t}dt \\
\rightarrow [\hat{h}(-\omega)]^{*} &= \int_{-\infty}^{\infty}h(t)e^{-i \omega t}dt \\
\rightarrow [\hat{h}(-\omega)]^{*} &= \hat{h}(\omega)
\end{align}

This is Hermitian symmetry. In the case of a complex function of time $h(t) \equiv h_{+}(t) - i h_{\times}(t)$ the same sequence of operations yields


\begin{align}
\hat{h}(\omega) &= \int_{-\infty}^{\infty}[h_{+}(t) - i h_{\times}(t)] e^{-i \omega t} dt \\
\rightarrow [\hat{h}(\omega)]^{*} &= \int_{-\infty}^{\infty}[h_{+}(t) + i h_{\times}(t)]e^{i \omega t}dt \\
\rightarrow [\hat{h}(-\omega)]^{*} &= \int_{-\infty}^{\infty}[h_{+}(t) + i h_{\times}(t)]e^{-i \omega t}dt \\
\rightarrow [\hat{h}(-\omega)]^{*} &= \widehat{h^{*}}(\omega)
\end{align}

The right hand side of the last formula is the \textit{fourier transform of the complex conugate of}$h(t)$. The "Complex conjugate in time" operation is defined as 

\begin{align}
\mathcal{I}\hat{h}(\omega) &= \int_{-\infty}^{\infty}h^{*}(t)e^{-i \omega t}dt 
\end{align}

This is also the fourier transform of the couplex conjugate of the time domain function $h(t)$, and therefore equivalent, for any complex function, to:

\begin{align}
\mathcal{I}\hat{h}(\omega) = [\hat{h}(-\omega)]^{*}
\end{align}

This is true for any arbitrary function that can be written $h(t) = h_{+}(t) + i h_{\times}(t)$. 

\end{document}
