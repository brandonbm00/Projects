%% This is an example first chapter.  You should put chapter/appendix that you
%% write into a separate file, and add a line \include{yourfilename} to
%% main.tex, where `yourfilename.tex' is the name of the chapter/appendix file.
%% You can process specific files by typing their names in at the 
%% \files=
%% prompt when you run the file main.tex through LaTeX.
\chapter{Bayesian Statistics}


\section{Bayesian Statistics}
We select a Bayesian posterior probability as our main analysis technique as its ability to quantify and propagate uncertainties in parameter estimates is well documented \cite{Volume-2} as a tool of choice for gravitational wave signal processing. Bayes' Theorem follows from the symmetric definition of conditional probability depicted visually in figure \ref{fig:distrib}. Let the joint probability evaluated at any point $(x,y)$ in the plane be denoted $P(x,y)$. We may take a cross section through the joint probability distribution along any direction we choose, let us call such a cross section parallel to the $x$ axis the conditional probability of $x$ given $y$, and denote it $P(x|y)$. A perpandicular cross section would simply be $P(y|x)$. We see that we can compute $P(x,y)$ in two equivalent ways: scaling $P(x|y)$ by $P(y)$ or by scaling $P(y|x)$ by $P(x)$. Thus we have that 


\begin{figure}[h!]
\center
\includegraphics[trim={1cm, 1cm, 1cm, 1cm}, clip]{biviariate_norm_done.eps}
\caption{A joint probability distribution $P(x,y)$. The contours parallel with the $x$ direction represent different $P(y|x)$.}
\label{fig:sphere}
\end{figure}

\begin{align} \label{eq:bayes}
P(x|y)P(y) &= P(y|x)P(x) \\
P(x|y) &= \frac{P(y|x)P(x)}{P(y)}
\end{align}

Which is Bayes' theorem. The denominator of the expression on the right hand side is often rewritten as a sum over all possible situations in which $y$ is an outcome. Reading $\neg x$ as "not x", we have

\begin{align}\label{eq:continuous}
P(y) &= P(y|x)P(x) + P(y|\neg x)P(\neg x) \\
&= \sum_{k=1}^K P(y|x_k)P(x_k) 
\end{align}


\clearpage 
This allows us to rewrite the posterior probability as

\begin{align}
P(x|y) = \frac{P(y|x) P(x)}{\sum_{k=1}^K P(y|x_k)P(x_k)}
\end{align}

Multiplying the right hand side of this equation by the quantity $\frac{1}{P(x)P(y|\neg x)}$ one has 

\begin{align} \label{eq:lratio}
P(x|y) &= \frac{\Lambda(x|y)}{\Lambda(x|y) + \frac{P(\neg x)}{P(x)}}
\end{align}

Where we have used the \textit{likelihood ratio} $\Lambda(x|y) \equiv \frac{P(y|x)}{P(y|\neg x)}$ to simplify the expression. The likelihood ratio has a simple interpretation, it is simply the ratio of the probability of event $y$ occuring in the case event $x$ occurs and the probability of event $y$ occuring given that event $x$ \textit{does not} occur. In fact, this form of Bayes theorem indicates that the posterior probability for $x$ given $y$ is directly proportional to the likelihood ratio. This is particularly important in the Bayesian treatment of gravitational waves where the central question for study is the probability of observing the data in question given that there may or may not be a signal present.  

\subsection{Signal Processing}
Noise in the LIGO detectors is modeled as \textit{Gaussian Noise}, which describes a length $N$ set of time series samples $\{x_j\}$ with the following probability of observation:

\begin{align}
P(\{x_j\}) = \left[\frac{1}{\sqrt{2 \pi} \sigma}\right]^N e^{-\frac{1}{2 \sigma^2} \sum_{j = 0}^{N - 1}x_j^2}
\end{align}

If we allow the $\{x_j\}$ to be weighted by the noise sensitivity in frequency space (also known as the \textit{power spectral density}) $S(f) = \sum_{k=0}^K n^*(f_k)n(f_k) \equiv \braket{n(f)|n(f)}$ then we may write the probability density for any Gaussian noise process as 

\begin{align}
P_x(\{x(t)\}) \propto e^{-\frac{1}{2}\braket{x(t)|x(t)}}
\end{align}

For gravitational wave signal processing we are interested in evaluating the equivalent of the likelihood ratio  in equation \ref{eq:lratio} for the null and signal present hypotheses, $\mathcal{H}_0$ and $\mathcal{H}_1$ respectively. Following the treatment outlined in literature \cite{rapidpe} This is

\begin{align}
\Lambda(\mathcal{H}_1|d) &= \frac{P(d|\mathcal{H}_1)}{P(d|\mathcal{H}_0)}
\end{align}

In the above equation $\mathcal{H}_0$ is in direct correspondence with $\neg x$ and $\mathcal{H}_1$ is in direct correspondence with $x$. If there is no signal present in the data, then the data and noise are equal, meaning 

\begin{align}
d(t) &= n(t) \\
P(d(t)|\mathcal{H}_0) &\propto e^{-\frac{1}{2}\braket{d(t)|d(t)}}
\end{align} 

Whereas if there is a signal present in the data, then the noise is the \textit{residual} signal after the gravitational wave potion of the signal $h(t)$ is subtracted from the total signal $d(t)$: 

\begin{align}
d(t) &= n(t) + h(t) \\
p(d(t)|\mathcal{H}_1) &\propto e^{-\frac{1}{2}\braket{d(t) - h(t)|d(t) - h(t)}}
\end{align}

So the likelihood ratio for gravitational wave signal processing is exactly equal to 

\begin{align}
\Lambda(\mathcal{H}_1|d(t)) &= \frac{e^{-\frac{1}{2}\braket{h(t)-d(t)|h(t)-d(t)}}}{e^{-\frac{1}{2}\braket{d(t)|d(t)}}}
\end{align}

By equation \ref{eq:bayes}  the posterior probability of a gravitational wave signal with parameters $\mu$ is given by 

\begin{align}\label{eq:parampos}
P[\vec{\mu}, \mathcal{H}_1|d(t)] = \frac{P[d(t)|\vec{\mu}, \mathcal{H}_1]P(\vec{\mu})}{P[d(t), \mathcal{H}_1]}
\end{align}

Extending equation \ref{eq:continuous} to the case of a continuous state space we have for the denominator

\begin{align}
P[d(t), \mathcal{H}_1] &= \int_{\Omega}P[d(t)|\vec{\mu}, \mathcal{H}_1]P(\vec{\mu})d\vec{\mu}
\end{align}

Where $\Omega$ indicates integration over the entire probabilistic state space. Dividing both the top and bottom of the right hand side of equation \ref{eq:parampos} by $P(d(t)|\vec{\mu}, \mathcal{H}_1)$ gives us 

\begin{align}\label{eq:bayeslikely}
P(\vec{\mu}, \mathcal{H}_1|d(t)) = \frac{\Lambda(d(t)|\vec{\mu}, \mathcal{H}_1)P(\vec{\mu})}{\int_{\Omega}\Lambda(d(t)|\vec{\mu}, \mathcal{H}_1)P(\vec{\mu})d\vec{\mu}}
\end{align}

Note that the integral in the denominator may be factored into components that are functions of independent groups of parameters. For the purposes of gravitational wave data analysis it is factored into intrinsic and extrinsic parameters, the former (masses, spins) being responsible for the actual orbital dynamics of the system, and the latter (distance, inclination, right ascension, declination) depending only on the nature of the observer relative to the source. We label these sets of paramters $\vec{\lambda}$ and $\vec{\theta}$ respectively. The denominator of equation \ref{eq:bayeslikely} is thus factorizable as 

\begin{align}
\int_{\Omega}\Lambda(d(t)|\vec{\mu}, \mathcal{H}_1)d\vec{\mu} &= \int_{\Omega}\Lambda(d(t)|\vec{\lambda}, \mathcal{H}_1)P(\vec{\lambda})d\vec{\lambda}\int_{\Omega}\Lambda(d(t)|\vec{\theta}, \mathcal{H}_1)P(\vec{\theta})d\vec{\theta} 
\end{align} 

Note that while evaluating the likelihood for a single set of parameters in the numerator of the above equation may not be any serious challenge, computing the integral in the denominator is. Many schemes \cite{rapidpe} \cite{lalinference} \cite{mcmc} have been developed to tackle this particular problem and generally involve simulating the posterior through MCMC sampling. These techniques are highly developed and at the time of writing can provide converged results on the order of days, however they are limited in their parallelism due fundamental serialization of the algorithm. Cutting edge implementations involving parallel tempering and some forms of ensemble MCMC have shown promising results, however none can take advantage of the many thousands of cores availble across the LIGO data grid. Methods that can scale out to the computational resources provided by the LIGO data grid are needed to reduce runtimes to minutes or less.   

\subsection{Harmonic Mode Decomposition}

For a single detector $k$ We can write our gravitational wave signal $h_k(t) = h_{+,k}(t) - ih_{\times, k}(t)$ as a sum over products of harmonic mode strain values $h_{l,m}(\vec{\lambda})$ and corresponding spin-weighted spherical harmonics, scaled by the reference distance and antenna factor:

\begin{align}
h_k(t) = \frac{D_r}{D}F_k(\theta)\sum_{(l,m)}h_{k, (l,m)}(t, \vec{\lambda})Y_{(l,m)}(\vec{\theta})
\end{align}

Where $D_r$ is the luminosity reference distance to the binary. This is defined similarly to the manner in which apparent magnitude is defined versus absolute magnitude in astonomy. We substitute this expression for the signal into the expression for the likelihood ratio we find, taking the logarithm, that

\begin{align}
\mathcal{L} &= \prod_k e^{-\frac{1}{2}\braket{d - h_k|d-h_k} + \braket{d|d}} \\
\ln\mathcal{L} &= -\frac{1}{2}\sum_{k}\left[\braket{d - h_k|d-h_k} + \braket{d|d}\right] \\
&= \frac{1}{2}\sum_{k}\left[-\cancel{\braket{d|d}}+\braket{d|h_k}+\braket{h_k|d}+\braket{h_k|h_k} + \cancel{\braket{d|d}}\right] \\
\end{align}

Now, since for complex vector spaces $\braket{a|b} = \braket{b|a}*$, we have that $\braket{d|h_k} + \braket{h_k|d} = 2\Re\braket{h_k|d}$. Then, 

\begin{align}
\ln\mathcal{L} &= \frac{1}{2}\sum_{k}\left[2\braket{h_k|d} + \braket{h_k|h_k}\right] \\
&= \frac{1}{2} \sum_{k}\Re\left[\braket{2\frac{D_r}{D}F_k(\vec{\theta})\sum_{(l,m)}h_{k,(l,m)}(\lambda)Y_{(l,m)}(\vec{\theta})|d}\right] \\
&+ \left[\frac{D_r}{D}\right]^2F_k^2(\vec{\theta})\sum_{(l,m),(l',m')}h^*_{(l,m)}(\vec{\lambda})h_{(l',m')}(\vec{\lambda})Y^*_{(l,m)}(\vec{\theta})Y_{(l',m')}(\vec{\theta})
\end{align}  

Now, for any complex number $z = a + ib$, we have that $z*z + \Re(z^2) = 2z*z$, so $z*z = \frac{1}{2}(z*z + \Re z^2)$. Then the double sum over $l,m$ pairs becomes




The defining feature of this form of the likelihood is the separation between intrinsic and extrinsic parameters. The intrinsic parameters of the binary exist only within the $h_{k,(l,m)}$, whereas the parameters that depend on the observer are contained entirely within the antenna factor and spherical harmonics. Work by \cite{rapidpe} (RapidPE) has shown that this can be exploited to accelerate computation of the likelihood by precomputing the orbital dynamics of the binary and marginalizing over the extrinsic parameters. The same work computes the posterior over the \textit{intrinsic} parameter space by spreading a grid of points across the $\mathcal{M}, \eta$ plane and computing the integral at each such point, since each of these integrations is independent each point for which the value of the posterior is desired may be assigned its own computational resources. This enables the necessary scaling - the posterior between the points is simply interpolated after all the cores are done. 

\subsection{Precessional Effects}
This work aims to extend and accelerate the ability for RapidPE to estimate gravitational wave parameters. RapidPE is written in Python, and prior to this work significant attention was given to code optimization to obtain the maximum performance possible through Python. Since the terms of the likelihood that may be precomputed involve computationally expensive fourier transforms, calculation of the cross terms $V$ were approximated using the properties of fourier transform to relate 


