\chapter{Formulas and Derviations}\label{Appendix C}

\subsection{Christoffel Symbols}

The Christoffel Symbols are defined in terms of the metric tensor $g_{\mu \nu}$ as 

\begin{align}
\Gamma^{\rho}_{\mu \nu} = \frac{1}{2}g^{\sigma \rho}(\partial_{\nu}g_{\mu \rho} + \partial_{\mu}g_{\nu \rho} - \partial_{\sigma}g_{\mu \nu})
\end{align}

\clearpage
\newpage

\subsection{Details on Linearized Gravity}
Inserting the expanded form of the metric into the term within the brackets gives us six terms times two at the front, for twelve total. Thankfully, three correspond to the standard flat-space Christoffel symbols which can be made zero through a choice of coordinates (in fact, orthonormal Cartesian coordinates), three involve derivatives of $\eta_{\mu \nu}$ which are zero, and three are second order in $h_{\mu \nu}$, which we assume are zero. This leaves us with three:

\begin{align}\label{eq:linchrist}
\Gamma^{\mu}_{\alpha \beta} &= \frac{1}{2}\eta^{\mu \nu}\left[\partial_{\beta}h_{\alpha \nu} + \partial_{\alpha} h_{\beta \nu} - \partial_{\nu}h_{\alpha \beta}\right] \\
&= \frac{1}{2} \left[\partial_{\beta} h^{\mu}_{\alpha} + \partial_{\alpha}h_{\beta}^{\mu} - \partial^{\mu}h_{\alpha \beta}\right] 
\end{align} 

Where in the last term $\partial^{\mu} = \eta^{\mu \nu}\partial_{\nu}$ was used to raise the index on the actual partial derivative operator itself. It should be noted at this point that since we neglect terms that are second order in the Christoffel symbols themselves, in linearized gravity, the covariant derivative of a metric perturbation is equivalent to the normal partial derivate. The same assumption allows for simplification of the Reimann and Ricci tensors:

The christoffel symbols contain some information about the curvature of space as a function of the coordinates, and thus in the framework of linearized gravity products of Christoffel symbols are second order:   

\begin{align}
R^{\rho}_{\sigma \mu \nu} = \partial_{\mu}\Gamma^{\rho}_{\nu \sigma} - \partial_{\nu}\Gamma^{\rho}_{\mu \sigma} 
\end{align}

\begin{align}
R^{\rho}_{\mu \sigma \nu} &= \partial_{\sigma}\Gamma^{\rho}_{\mu \nu} - \partial_{\nu}\Gamma^{\rho}_{\mu \sigma} \\
R_{\mu \nu} = R^{\rho}_{\mu \rho \nu} &= \partial_{\rho}\Gamma^{\rho}_{\nu \mu} - \partial_{\mu}\Gamma^{\rho}_{\mu \rho}
\end{align}

Plugging in equation \ref{eq:linchrist} we have

\begin{align}
R_{\mu \nu} & = \frac{1}{2}\partial_{\alpha}\left[\partial_{\nu}h^{\alpha}_{\mu} + \partial_{\mu}h^{\alpha}_{\nu} - \partial^{\alpha}h_{\mu \nu}\right] - \frac{1}{2}\partial_{\nu}\left[\partial_{\alpha}h^{\alpha}_{\mu} + \partial_{\mu}h^{\alpha}_{\alpha} - \partial^{\alpha}h_{\mu \alpha}\right] \\
&= \frac{1}{2}\left[ \cancel{\partial_{\alpha}\partial_{\nu}h^{\alpha}_{\mu}} + \partial_{\alpha}\partial_{\mu}h^{\alpha}_{\nu} - \partial_{\alpha}\partial^{\alpha}h_{\mu \nu} - \cancel{\partial_{\nu}\partial_{\alpha}h^{\alpha}_{\mu}} - \partial_{\nu}\partial_{\mu}h^{\alpha}_{\alpha} + \partial_{\nu}\partial^{\alpha}h_{\mu \alpha} \right] \\
&= \frac{1}{2}\left[\partial_{\alpha}\partial_{\mu}h^{\alpha}_{\nu} + \partial_{\nu}\partial^{\alpha}h_{\mu \alpha} - \partial_{\alpha}\partial^{\alpha}h_{\mu \nu} - \partial_{\nu}\partial_{\mu}h^{\alpha}_{\alpha}\right]\label{eq:linricci}
\end{align}

The quantity $h^{\alpha}_{\alpha}$ is the trace of the pertubation metric and represents, in a sense, an overall measure of the \textit{strength} or \textit{amplitude} of the perturbation, this is often called just $h$. The term $\partial^{\alpha}\partial_{\alpha}$ is an inner product that when formed through the Minkowski metric becomes the wave operator, $\Box = -\partial_{t}^{2} + \nabla^2$. The Ricci tensor is obtained by contracting the Reimann tensor over the two free indices:

\begin{align}
R \equiv \eta^{\mu \nu}R_{\mu \nu} &= \frac{1}{2}\left[\partial_{\alpha}\partial^{\mu}h_{\mu}^{\alpha} + \partial_{\mu} \partial^{\alpha}h^{\mu}_{\alpha} - \Box h - \Box h\right]
\end{align}

Since $\alpha$ and $\mu$ are summed indices, the first two terms are the same. Dropping the $\frac{1}{2}$, the Ricci scalar curvature is 

\begin{align}
R = \partial_{\alpha}\partial_{\mu}h^{\mu \alpha} - \Box h
\end{align}

Many authors choose to simplify the expression for the Ricci Tensor by defining the quantity

\begin{align}\label{eq:lorentzgauge}
V_{\nu} = \partial_{\alpha}h^{\alpha}_{\nu} - \frac{1}{2}\partial_{\nu}h
\end{align}

With which the first two terms of the linearized Ricci tensor can be reorganized as

\begin{align}
\partial_{\mu}\partial_{\alpha}h_{\nu}^{\alpha} - \partial_{\mu}\partial_{\nu}h &= \partial_{\mu}\left[\partial_{\alpha}h_{\nu}^{\alpha} - \partial_{\nu}h\right] \\
&= \partial_{\mu}V_{\nu} - \frac{1}{2}\partial_{\mu}\partial_{\nu}h
\end{align}

Which upon substitution into equation \ref{eq:linricci} for $R_{\mu \nu}$ yields

\begin{align}
R_{\mu \nu} &= \frac{1}{2}\left[\partial_{\mu}V_{\nu} - \frac{1}{2}\partial_{\mu}\partial_{\nu}h + \partial^{\alpha}\partial_{\nu}h_{\mu \alpha} - \Box h_{\mu \nu}\right]
\end{align}

The two unsimplified terms form the quantity $\partial_{\nu}\left[\partial^{\alpha}h_{\mu \alpha} - \frac{1}{2}\partial_{\mu}h\right]$ which we recognize as $\partial_{\nu}V_{\mu}$, leaving us with

\begin{align}
R_{\mu \nu} = \frac{1}{2}\left[\partial_{\mu}V_{\nu} + \partial_{\nu}V_{\mu} - \Box h_{\mu \nu}\right] 
\end{align}

Setting the right hand side of this equation equal to zero gives us the linearized, vacuum Einstein Equation $R_{\mu \nu} = 0$:

\begin{align}
\frac{1}{2}\left[\partial_{\mu}V_{\nu} + \partial_{\nu}V_{\mu} - \Box h_{\mu \nu}\right] = 0 
\end{align}

$h_{\mu \nu}$ is a two index object in a four dimensional spacetime, which in the general case would present with sixteen independent components each of which would need to be explicitly determined to form a complete picture of the physics. However in our specific case it will turn out that there are far fewer real independent components then this, which can be observed by chiseling away at the above expression in the following manner. Firstly, the metric of general relativity is taken to be symmetric, because if we assume that the tensor $dx^{\mu}dx^{\nu}$ is symmetric (i.e. $dx^{\nu}$ and $dx^{\nu}$ commute) then only the symmetric part of any $g_{\mu \nu}$ contributes when the quantity $g_{\mu \nu}dx^{\mu}dx^{\nu}$ is computed. Thus we have 

\begin{align}
h_{\mu \nu} = h_{\nu \mu}
\end{align}

Which, for a sixteen component tensor, makes six of the components redundant. This leaves us with ten. Furthermore, there is \textit{gauge freedom} present in the linearized vacuum Einstein equation. This is best illustrated with reference to the equations of classical electrodynamics, which describe the motion of charged particles subject to the electric and magnetic vector potentials, $V$ and $\vec{A}$ respectively. It is somewhat easily demonstrable that the same equations of motion for electrically charged particles are unchanged when $V$ and $\vec{A}$ are subject to transformations of the form 

\begin{align}\label{eq:emtrans}
\vec{A} &\rightarrow \vec{A} + \nabla \Psi \\
V &\rightarrow V -\frac{\partial \Psi}{\partial t}
\end{align}  

The function $\Psi$ is known as the \textit{gauge function} and it can be absolutely anything we want, as long as we adjust the potentials accordingly. Within such a framework, one can impose certain conditions on the potentials that simplify the mathematics, such as requiring that 

\begin{align}\label{eq:emgauge}
\nabla \cdot \vec{A} + \frac{1}{c}\frac{\partial V}{\partial t} &= 0 \\
\rightarrow \partial^{\mu}A_{\mu} &= 0
\end{align}

It is somewhat counterintuitive why imposing this particular condition on $\vec{A}$ and $V$ themselves is equivalent to transforming the potentials via the gauge function, but indeed it is, it is just hidden! While we have made no reference to $\Psi$, we have \textit{implicitly} assumed that there is such a $\Psi$ that will \textit{make} the restriction on $\vec{A}$ and $V$ true. It actually is there inside the $V$ and $\vec{A}$ written in equation \ref{eq:emgauge}. We do not have to state \textit{what it is}, as long as we know that it exists. A proof that there legitimately is such a gauge function exists but is out of the scope of this outline. 
There is a one-to-one correspondance of this formalism with the equivalent gauge in general relativity, which is known as the Lorentz gauge. In the same sense as equation \ref{eq:emtrans}, it turns out that the equations of motion for massive particles in a slightly perturbed spacetime are invariant under coordinate transformations of the form  

\begin{align}
x^{\mu} \rightarrow x'^{\mu} - \xi^{\mu}
\end{align}

In particular, we can show that there exists a coordinate transformation such that 

\begin{align}
\partial_{\mu}V_{\nu} = \partial_{\mu}\partial_{\alpha}h^{\alpha}_{\nu} - \frac{1}{2}\partial_{\mu}\partial_{\nu}h = 0
\end{align} 

\textbf{PUT PROOF HERE}

\begin{align}
\text{PUT PROOF HERE}
\end{align}

Thus coordinate transformations that obey $\Box \xi = 0$ will allow us to impose the Lorentz gauge. Within this gauge, the linearized, vacuum Einstein equation becomes simply

\begin{align}\label{eq:flteinst}
\Box h_{\mu \nu} = 0
\end{align}

This is a four dimensional wave equation for the remaining components of $h_{\mu \nu}$. Moreover, we have shown that under coordinate transformations that do not change the underlying physics, there exists an explicit relationship between components of the metric given by equation \ref{eq:lorentzgauge}. In the same way that symmetry shows that six of the metric components to be functions of the others, The index $\nu$ indicates that in four dimensions there are four additional explicit relationships between metric components that correspond to a loss of four additional degrees of freedom. This leaves us with six.  




In flat space, equation \ref{eq:flteinst} for some arbitrary function $f(x)$ in place of the $h_{\mu \nu}$ would read

\begin{align}
\Box f(x) = \eta_{\alpha \beta} \frac{\partial^2 f}{\partial x^{\alpha} \partial x^{\beta}} &= 0\\
-\frac{\partial^2 f}{\partial t^2} + \nabla^2f &= 0
\end{align} 

This is the general form of the time dependent wave equation whose solutions take the form

\begin{align}
f(x) = a e^{i \mathbf{k} \cdot \mathbf{x}}
\end{align}

Where $\mathbf{x}$ is the standard positon four-vector and $\mathbf{k}$ is the four-wave-vector. Expanding the inner product in the exponential again using the Minkowski metric as $-k^t t + \vec{k}\cdot \vec{x}$ and plugging the result into equation \ref{eq:flteinst} we have 

\begin{align}
\text{PUT EQ HERE}
\end{align}




