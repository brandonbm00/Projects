% $Log: abstract.tex,v $
% Revision 1.1  93/05/14  14:56:25  starflt
% Initial revision
% 
% Revision 1.1  90/05/04  10:41:01  lwvanels
% Initial revision
% 
%
%% The text of your abstract and nothing else (other than comments) goes here.
%% It will be single-spaced and the rest of the text that is supposed to go on
%% the abstract page will be generated by the abstractpage environment.  This
%% file should be \input (not \include 'd) from cover.tex.
The September 14th, 2015 detection of gravitational waves (GW) by the Advanced LIGO detectors in Hanford, Washington and Livingston, Louisiana has galvanized the GW astronomy community and ushered in a new and unexplored method of observing astrophysical phenomena. With it comes the opportunity to couple GW signals with electromagnetic (EM) observations in a manner that they may inform each other, maximizing the science returns from both types of detections. To that end it is imperative that GW signal processing techniques be developed to such a point that they may rapidly and accurately report system parameters from possible progenitors of EM signals of interest. In particular the opportunity to rapidly localize and infer crucial system parameters of Gamma Ray Burst (GRB) and Kilonova progenitors hinges on prompt reporting of the sky location and orbital angular momentum direction of compact binary coalescences (CBC). These systems are thought to be the most likely sources of detectable GW signals in the Advanced LIGO era. This thesis presents a vectorization and subsequent GPU implementation of a high performance parameter estimation algorithm which reports a Bayesian posterior probability over the CBC parameter space, given an data time series thought to contain a GW signal. We report on the speedup of the algorithm and its role in extending the capability of RapidPE to perform low latency electromagnetic follow-up as it pertains to GRBs and Kilonova afterglows.
