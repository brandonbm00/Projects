%%%%%%%%%%%%%%%%%%%%%%%%%%%%%%%%%%%%%%%%%
% Short Sectioned Assignment
% LaTeX Template
% Version 1.0 (5/5/12)
%
% This template has been downloaded from:
% http://www.LaTeXTemplates.com
%
% Original author:
% Frits Wenneker (http://www.howtotex.com)
%
% License:
% CC BY-NC-SA 3.0 (http://creativecommons.org/licenses/by-nc-sa/3.0/)
%
%%%%%%%%%%%%%%%%%%%%%%%%%%%%%%%%%%%%%%%%%

%----------------------------------------------------------------------------------------
%	PACKAGES AND OTHER DOCUMENT CONFIGURATIONS
%----------------------------------------------------------------------------------------

\documentclass[paper=a4, fontsize=11pt]{scrartcl} % A4 paper and 11pt font size

\usepackage[T1]{fontenc} % Use 8-bit encoding that has 256 glyphs
\usepackage[adobe-utopia]{mathdesign}
%\usepackage{fourier} % Use the Adobe Utopia font for the document - comment this line to return to the LaTeX default
\usepackage[english]{babel} % English language/hyphenation
\usepackage{amsmath,amsfonts,amsthm} % Math packages

\usepackage{lipsum} % Used for inserting dummy 'Lorem ipsum' text into the template

\usepackage{listings}

\usepackage{sectsty} % Allows customizing section commands

\usepackage{braket}

\allsectionsfont{\centering \normalfont\scshape} % Make all sections centered, the default font and small caps


\usepackage{fancyhdr} % Custom headers and footers
\pagestyle{fancyplain} % Makes all pages in the document conform to the custom headers and footers
\fancyhead{} % No page header - if you want one, create it in the same way as the footers below
\fancyfoot[L]{} % Empty left footer
\fancyfoot[C]{} % Empty center footer
\fancyfoot[R]{\thepage} % Page numbering for right footer
\renewcommand{\headrulewidth}{0pt} % Remove header underlines
\renewcommand{\footrulewidth}{0pt} % Remove footer underlines
\setlength{\headheight}{13.6pt} % Customize the height of the header

\numberwithin{equation}{section} % Number equations within sections (i.e. 1.1, 1.2, 2.1, 2.2 instead of 1, 2, 3, 4)
\numberwithin{figure}{section} % Number figures within sections (i.e. 1.1, 1.2, 2.1, 2.2 instead of 1, 2, 3, 4)
\numberwithin{table}{section} % Number tables within sections (i.e. 1.1, 1.2, 2.1, 2.2 instead of 1, 2, 3, 4)

\setlength\parindent{0pt} % Removes all indentation from paragraphs - comment this line for an assignment with lots of text

%----------------------------------------------------------------------------------------
%	TITLE SECTION
%----------------------------------------------------------------------------------------

\newcommand{\horrule}[1]{\rule{\linewidth}{#1}} % Create horizontal rule command with 1 argument of height

\title{	
\normalfont \normalsize 
\textsc{Northwestern University} \\ [25pt] % Your us get downniversity, school and/or department name(s)
\horrule{0.5pt} \\[0.4cm] % Thin top horizontal rule
\huge Single Detector Log Likelihood \\ % The assignment title
\horrule{2pt} \\[0.5cm] % Thick bottom horizontal rule
}

\author{Brandon B. Miller} % Your name

\date{\normalsize\today} % Today's date or a custom date

\begin{document}

\maketitle % Print the title

%----------------------------------------------------------------------------------------
%	PROBLEM 1
%----------------------------------------------------------------------------------------
\subsection{Vectorization}
\renewcommand*{\arraystretch}{0.5}
Equation 24 from Arxiv $15502.05370v1.pdf$ for the network log likelihood reads 

\begin{equation}
\begin{align}
\ln{\mathcal{L}} = &\frac{D_{ref}}{D}Re \sum_{k}\sum_{(l,m)}\left[F_k Y_{lm}\right]^{*}Q_{k,lm} \\ 
& - \left[\frac{D_{ref}}{2D}\right]^{2}\sum_{k}\sum_{(l,m),(l',m')}\left[|F_k|^2 Y_{l,m}^{*}Y_{l',m'}U_{k,(l,m),(l'm')}\right] \\
 & - \left[\frac{D_{ref}}{2D}\right]^{2}\sum_{k}\sum_{(l,m),(l',m')}Re\left[  F_{k}^{2}Y_{l,m}Y_{l'm'}V_{k,(l,m),(l'm')}\right]
\end{align}
\end{equation}

Consider a single detector, thus dropping the sum over $k$. The first term is of the form $\vec{A}\cdot\vec{B} = \sum_{i=0}^{d}A_iB_i$, so if $Q_{k,(l,m)}$ were a simple vector, we could write it as

\begin{equation}
- \left[\frac{D_{ref}}{2D}\right]^{2}\sum_{k}\sum_{(l,m),(l',m')}\left[|F_k|^2 Y_{l,m}^{*}Y_{l',m'}U_{k,(l,m),(l'm')}\right] = - \left[\frac{D_{ref}}{2D}\right]^{2} F*\vec{Y}\cdot\vec{Q}
\end{equation}

However the $Q_{k,(l,m)}$ are actually harmonic mode time series and not single values. We desire a vector whose values are the likelihoods at each point in the time series. 

\newpage
Consider the case where we have only the $(2,-2), (2,0)$ and $(2,2)$ modes. If we write all the mode time series $Q^0, Q^1, Q^2...$ as the columns of a matrix , then the desired result is obtained with

\begin{equation}
F*
\begin{bmatrix}
Q^0_{2,-2} & Q^0_{2,+0} & Q^0_{2,+2} \\
Q^1_{2,-2} & Q^1_{2,+0} & Q^1_{2,+2} \\ 
Q^2_{2,-2} & Q^2_{2,+0} & Q^2_{2,+2} \\
\vdots & \vdots & \vdots
\end{bmatrix}
\begin{bmatrix}

\left(Y_{2,-2}\right)\\
\hspace{0mm} \\
\left(Y_{2,+0}\right) \\
\hspace{0mm} \\
\left(Y_{2,+2}\right) \\

\end{bmatrix}
=
\begin{bmatrix}
Q^0_{2,-2} Y_{2,-2} + Q^0_{2,-2}Y_{2,+0} + Q^0_{2,+2}Y_{2,+2} \\
Q^1_{2,-2} Y_{2,-2} + Q^1_{2,-2}Y_{2,+0} + Q^1_{2,+2}Y_{2,+2} \\
Q^2_{2,-2} Y_{2,-2} + Q^2_{2,-2}Y_{2,+0} + Q^2_{2,+2}Y_{2,+2} \\
\vdots
\end{bmatrix}

\end{equation} 

With $\vec{Y}$ and $\mathbf{Q}$ defined as the matrix and vector above respectively, we have for the first term

\begin{equation}
\frac{D_{ref}}{D}Re\left[\mathbf{Q}\left(F\vec{Y}\right)^{*}\right]
\end{equation}

The second term is a sum once over all the possible combinations of $(l,m), (l',m')$ pairs using the $U_{(l,m),(l',m')}$ cross terms. Its result is a scalar quantity made up of terms like

\begin{align}
&Y_{2,-2}^{*}Y_{2,-2}U_{(2,-2),(2,-2)} + Y_{2,-2}^{*}Y_{2,+0}U_{(2,-2),(2,+0)} + Y_{2,-2}^{*}Y_{2,+2}U_{(2,-2),(2,+2)} \\ 
 +  \ &Y_{2,+0}^{*}Y_{2,-2}U_{(2,+0),(2,-2)} + Y_{2,+0}^{*}Y_{2,+0}U_{(2,+0),(2,+0)} + Y_{2,+0}^{*}Y_{2,+2}U_{(2,+0),(2,+2)} \\ 
+ \  &Y_{2,+2}^{*}Y_{2,-2}U_{(2,+2),(2,-2)} + Y_{2,+2}^{*}Y_{2,+0}U_{(2,+2),(2,+0)} + Y_{2,+2}^{*}Y_{2,+2}U_{(2,+2),(2,+2)}
\end{align}


If we pack the $U_{(l,m),(l',m')}$ into the matrix $\mathbf{U}$ as defined below then the following set of matrix operations produces the same sum

\begin{align*}
\begin{bmatrix}
Y_{2,-2}^{*} &Y_{2,+0}^{*}  &Y_{2,+2}^{*}   
\end{bmatrix}
\begin{bmatrix}
U_{(2,-2),(2,-2)} &  U_{(2,-2),(2,+0)} &  U_{(2,-2),(2,+2)} \\
U_{(2,+0),(2,-2)} &  U_{(2,+0),(2,+0)} &  U_{(2,+0),(2,+2)} \\
U_{(2,+2),(2,-2)} &  U_{(2,+2),(2,+0)} &  U_{(2,+2),(2,+2)}
\end{bmatrix}
\begin{bmatrix}
Y_{2,-2} \\
Y_{2,+0} \\
Y_{2,+2}
\end{bmatrix}
\end{align*}

because when you multiply $\mathbf{U}$ into $\vec{Y}$ this simplifies to 



\begin{align*}
\begin{bmatrix}
Y_{2,-2}^{*} &Y_{2,+0}^{*}  &Y_{2,+2}^{*}   
\end{bmatrix}
\begin{bmatrix}
U_{(2,-2),(2,-2)} Y_{2,-2} + U_{(2,-2),(2,-2)}Y_{2,-2} + U_{(2,-2),(2,-2)}Y_{2,-2} \\ 
U_{(2,-2),(2,-2)} Y_{2,-2} + U_{(2,-2),(2,-2)}Y_{2,-2} + U_{(2,-2),(2,-2)}Y_{2,-2} \\ 
U_{(2,-2),(2,-2)} Y_{2,-2} + U_{(2,-2),(2,-2)}Y_{2,-2} + U_{(2,-2),(2,-2)}Y_{2,-2} 
\end{bmatrix}
\end{align*}

Which becomes the desired scalar. This allows us to write the second term as

\begin{align}
- \left[\frac{D_{ref}}{2D}\right]^{2}\sum_{k}\sum_{(l,m),(l',m')}\left[|F_k|^2 Y_{l,m}^{*}Y_{l',m'}U_{k,(l,m),(l'm')}\right] =  - \left[\frac{D_{ref}}{2D}\right]^{2} |F^2|\vec{Y}^{*}\mathbf{U}\vec{Y}
\end{align}

We must always set up the spherical harmonic vectors based on the value of $m$ and the cross terms in row-major form based first on $m_2$ and then on $m_1$. If we organize the matrix $\mathbf{V}$ in the same way then the same set of steps will lead us to conclude that  

\begin{align}
- \left[\frac{D_{ref}}{2D}\right]^{2}\sum_{k}\sum_{(l,m),(l',m')}Re\left[  F_{k}^{2}Y_{l,m}Y_{l'm'}V_{k,(l,m),(l'm')}\right] = - \left[\frac{D_{ref}}{2D}\right]^{2}Re \left[F^2 \vec{Y}\mathbf{V}\vec{Y} \right]
\end{align}

Combining the results the single detector log likelihood is 

\begin{align}
\ln{\mathcal{L}} = \frac{D_{ref}}{D}\Re\left[\mathbf{Q}\left(F\vec{Y}\right)^{*}\right] - \left[\frac{D_{ref}}{2D}\right]^{2}\left[|F|^2 \vec{Y}^{*}\mathbf{U}\vec{Y} - \Re\left(F^2 \vec{Y}\mathbf{V}\vec{Y}\right) \right]
\end{align}

\subsection{Implementation}

As it stands the data is organized into dictionaries keyed by tuples corresponding to modes or pairs of modes. I wrote a very basic function that takes one of these dictionaries as argument and produces matrices organized as described in the previous section. I do not think it requires a huge amount of attention since it just needs to run once at startup.

\vspace{10mm}

I have written only the most basic implementation for this that I can think of: 

\vspace{10mm}

\begin{lstlisting}[language=Python]

def sdll_matrix(rholm_vals, crossTermsU, crossTermsV, Ylms, F, dist):
 #Compute single detector log likelihood with vectorized operations
 invDistMpc = distRef/dist
 Fstar = np.conj(F)

 Ylms_conj = np.conj(Ylms)
 term1 = Fstar*np.dot(rholm_vals, Ylms_conj)
 term2 = np.dot(Fstar*F*Ylms_conj, np.dot(crossTermsU, Ylms))
 term3 = np.real(np.dot(F*F*Ylms, np.dot(crossTermsV, Ylms)))
 return invDistMpc*np.real(term1) - 0.25*invDistMpc**2 * np.real((term2 + term3))

\end{lstlisting}


\end{document}
